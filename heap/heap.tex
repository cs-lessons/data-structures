\documentclass[11pt]{book}
\usepackage[margin=0.5in]{geometry}
\usepackage{makecell}
\usepackage{tikz}
\usetikzlibrary{graphdrawing.trees}

\begin{document}
	\setcounter{chapter}{3}
	\chapter{Heaps}
	\section{Introduction}
		The \textbf{heap} is a data structure typically used to keep track
		of the max or minimum of a set of elements. They are typically
		used when implementing priority queues because they can keep
		track of the highest (or lowest) priority easily and efficiently.
		As such, the only real similarity I can draw between the data structure
		and its name is the appearance of its representation.

	\section{Description}
		Because the \textbf{heap} is typically drawn as a tree, it will use similar
		terminology that can be found in the binary search tree pdf. \\

		\noindent The one relationship of a \textbf{heap} is that the parent node is
		always greater (or less depending on the kind of \textbf{heap}) than its
		children. There is no strict relationship between any other nodes aside
		from the fact that the relationship is transitive and it could be said
		that a descendant node will be less than its ancestor nodes. \\

		\noindent When drawn as a tree, a \textbf{heap} should always be a complete
		binary tree. If you don't remember what that means, it means everything is
		filled out except for the last level although that level must be filled
		starting from the left. This structure is something that will remaing even
		after insertion and deletion. For example, a \textbf{max heap} could look
		like this:

		\begin{center}
			\begin{tikzpicture}[nodes={draw, circle}, ->,
				level 1/.style={sibling distance=30mm},
				level 2/.style={sibling distance=15mm}]
				\node{100}
					child{ node{99}
						child{ node{7} }
						child{ node{88} } }
					child{ node{98}
						child{ node{4} }
						child[ missing ] };
			\end{tikzpicture}
		\end{center}

		\noindent As we can see, for any pair of parent-child nodes, the parent's value
		is always greater than the child's. For all the future examples, we'll be
		using a \textbf{max heap}.

	\section{Searching}
		Remember that a the only relationship that exists in a \textbf{heap} is the
		one between a parent and child node. How does this affect searching? At a
		glance, we can easily find the max because it is the root, but what
		about other values? You will actually have to traverse most of the tree
		to determine whether a specific value exists or not, so there is not a
		benefit to searching for a specific value.

	\section{Insertion}
		Ok, remember that we want our \textbf{heap} to stay as a complete binary
		tree. So where should we add a node to maintain this? At the lowest level as
		far left as we can. Consider the previously shown \textbf{heap}:

		\begin{center}
			\begin{tikzpicture}[nodes={draw, circle}, ->,
				level 1/.style={sibling distance=30mm},
				level 2/.style={sibling distance=15mm}]
				\node{100}
					child{ node{99}
						child{ node{7} }
						child{ node{88} } }
					child{ node{98}
						child{ node{4} }
						child[ missing ] };
			\end{tikzpicture}
		\end{center}

		\noindent Let's insert 1 into the \textbf{heap}. It will go on the lowest
		level in the first free spot.

		\begin{center}
			\begin{tikzpicture}[nodes={draw, circle}, ->,
				level 1/.style={sibling distance=30mm},
				level 2/.style={sibling distance=15mm}]
				\node{100}
					child{ node{99}
						child{ node{7} }
						child{ node{88} } }
					child{ node{98}
						child{ node{4} }
						child{ node{1} } };
			\end{tikzpicture}
		\end{center}

		\noindent Well that was easy. But wait, what if we had to insert a number
		that will end up breaking the parent child relationship? We just have to do
		a few extra steps; let's try inserting 101 this time.

		\begin{center}
			\begin{tikzpicture}[nodes={draw, circle}, ->,
				level 1/.style={sibling distance=30mm},
				level 2/.style={sibling distance=15mm}]
				\node{100}
					child{ node{99}
						child{ node{7}
							child{ node{101} }
							child[ missing ] }
						child{ node{88} } }
					child{ node{98}
						child{ node{4} }
						child{ node{1} } };
			\end{tikzpicture}
		\end{center}

		\noindent Alright, so now we notice the relationship is broken so let's
		just swap 101 and 7. What an easy fix!

		\begin{center}
			\begin{tikzpicture}[nodes={draw, circle}, ->,
				level 1/.style={sibling distance=30mm},
				level 2/.style={sibling distance=15mm}]
				\node{100}
					child{ node{99}
						child{ node{101}
							child{ node{7} }
							child[ missing ] }
						child{ node{88} } }
					child{ node{98}
						child{ node{4} }
						child{ node{1} } };
			\end{tikzpicture}
		\end{center}

		\noindent Uh oh, the relationship is still not satisfied! (99 < 101). Just keep
		swapping upwards until you're good!

		\begin{center}
			\begin{tabular*}{0.75\textwidth}{@{\extracolsep{\fill}}cc}
				\begin{tikzpicture}[nodes={draw, circle}, ->,
					level 1/.style={sibling distance=30mm},
					level 2/.style={sibling distance=15mm}]
					\node(c){100}
						child{ node(b){99}
							child{ node(a){101}
								child{ node{7} }
								child[ missing ] }
							child{ node{88} } }
						child{ node{98}
							child{ node{4} }
							child{ node{1} } };
					\path[]
						(a) edge[bend left] node[left, draw=none] {} (b)
						(b) edge[bend left] node[left, draw=none] {} (c);
				\end{tikzpicture} &
				\begin{tikzpicture}[nodes={draw, circle}, ->,
					level 1/.style={sibling distance=30mm},
					level 2/.style={sibling distance=15mm}]
					\node(c){101}
						child{ node(b){100}
							child{ node(a){99}
								child{ node{7} }
								child[ missing ] }
							child{ node{88} } }
						child{ node{98}
							child{ node{4} }
							child{ node{1} } };
				\end{tikzpicture}
			\end{tabular}
		\end{center}

		\noindent This movement of the node upwards is commonly called \textbf{bubbling up}.

	\section{Deletion}
		Deletion is only a tad more difficult. Unlike other data structures, when
		you delete from a \textbf{heap} (remove would be a better word), you always
		delete the root. Let's try deleting from the previous example:
		
		\begin{center}
			\begin{tikzpicture}[nodes={draw, circle}, ->,
				level 1/.style={sibling distance=30mm},
				level 2/.style={sibling distance=15mm}]
				\node(c){101}
					child{ node(b){100}
						child{ node(a){99}
							child{ node{7} }
							child[ missing ] }
						child{ node{88} } }
					child{ node{98}
						child{ node{4} }
						child{ node{1} } };
			\end{tikzpicture}
		\end{center}

		\noindent So we'll be deleting 101. But remember we want to keep the
		completeness of the tree! So let's take that last node and put it at the root.

		\begin{center}
			\begin{tikzpicture}[nodes={draw, circle}, ->,
				level 1/.style={sibling distance=30mm},
				level 2/.style={sibling distance=15mm}]
				\node(c){7}
					child{ node(b){100}
						child{ node(a){99} }
						child{ node{88} } }
					child{ node{98}
						child{ node{4} }
						child{ node{1} } };
			\end{tikzpicture}
		\end{center}

		\noindent We have a complete tree but something seems off... the relationship
		is broken again! This time, instead of \textbf{bubbling up}, we will
		\textbf{bubble down}. 7 is obviously out of place so we should move it down,
		but which way should it go? We want to satisfy the relationship for these 3 nodes
		so which of the following arrangements looks good?

		\begin{center}
			\begin{tabular*}{0.75\textwidth}{@{\extracolsep{\fill}}cccc}
				&
				\begin{tikzpicture}[nodes={draw, circle}, ->]
					\node{100}
						child{ node{7} }
						child{ node{98} };
				\end{tikzpicture} &
				\begin{tikzpicture}[nodes={draw, circle}, ->]
					\node{98}
						child{ node{100} }
						child{ node{7} };
				\end{tikzpicture} &
				&
			\end{tabular}
		\end{center}

		\noindent The left one does fixes everything, so it looks like the rule is swap
		with the maximum of the two children in this case!

		\begin{center}
			\begin{tikzpicture}[nodes={draw, circle}, ->,
				level 1/.style={sibling distance=30mm},
				level 2/.style={sibling distance=15mm}]
				\node{100}
					child{ node{7}
						child{ node{99} }
						child{ node{88} } }
					child{ node{98}
						child{ node{4} }
						child{ node{1} } };
			\end{tikzpicture}
		\end{center}

		\noindent We end up with this but we're still not done; I'll leave the rest as
		an exercise to the reader. Haha. Well obviously you swap 7 and 99 and then you're
		finished.

	\section{Array-Based Heap}
		Arrays... what? Interestingly enough, \textbf{heaps} are typically implemented in
		arrays instead of having node objects and linking them together with pointers!
		Let's go back to our previous \textbf{heap} (because it's easier to copy and
		paste than to make a new example):

		\begin{center}
			\begin{tikzpicture}[nodes={draw, circle}, ->,
				level 1/.style={sibling distance=30mm},
				level 2/.style={sibling distance=15mm}]
				\node{100}
					child{ node{99}
						child{ node{7} }
						child{ node{88} } }
					child{ node{98}
						child{ node{4} }
						child{ node{1} } };
			\end{tikzpicture}
		\end{center}

		\noindent All we have to do is dump these contents into an array left to right, top
		to bottom. We'll end up getting

		\[
			100 \rightarrow 99 \rightarrow 98 \rightarrow 7 \rightarrow 88 \rightarrow 4
			\rightarrow 1
		\]

		\noindent Let's try to find a way to determine the index of a child if we have the
		index of the parent. The index of 100 is 0, it's children have indicies 1 and 2;
		The index of 99 is 1, it's children have indicies 3 and 4; and so on After staring at
		the numbers multiple times, you might notice the following:
		\[
			parent = i; children = 2i + 1, 2i + 2
		\]
		So if you know the parent's index, you can get the children with this handy formula. And
		likewise
		\[
			child = i; parent = (i - 1) / 2
		\]
		But what about decimals? Use integer division!



	\section{Examples}
		Removed for now.

\end{document}
