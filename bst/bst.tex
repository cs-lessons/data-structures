\documentclass[11pt]{book}
\usepackage[margin=0.5in]{geometry}
\usepackage{makecell}
\usepackage{tikz}
\usetikzlibrary{graphdrawing.trees}

\begin{document}
	\setcounter{chapter}{2}
	\chapter{Binary Search Trees}
	\section{Introduction}
		The \textbf{binary search tree} is probably the first new concept that
		you will learn about. Fortunately, computer scientists are not very
		creative so the name somewhat hints at this data structure's use.
		Unfortunately, I cannot come up with a real world example of this.

	\section{Terminology}
		\begin{itemize}
			\item \textbf{node} \\
				A \textbf{binary search tree} is made up 0 or more \textbf{nodes}.
				You can imagine these as small containers with data along with
				two pointers (because this is a \textbf{binary} tree), typically
				denoted as left and right. In the following diagrams, a \textbf{node}
				will appear as a circle: \\
				\begin{center}
					\begin{tikzpicture}[nodes={draw, circle}, ->]
						\node{a};
					\end{tikzpicture}
				\end{center}
			\item \textbf{children} \\
				A node can have up to two \textbf{children} (again, \textbf{binary}).
				These may be referred to as the left and right \textbf{children}
				depending on what side they are on.
				\begin{center}
					\begin{tabular*}{0.75\textwidth}{@{\extracolsep{\fill}}ccc}
						\begin{tikzpicture}[nodes={draw, circle}, ->]
							\node{2}
								child{ node{1} }
								child[ missing ];
						\end{tikzpicture} &
						\begin{tikzpicture}[nodes={draw, circle}, ->]
							\node{2}
								child[ missing ]
								child{ node{3} };
						\end{tikzpicture} &
						\begin{tikzpicture}[nodes={draw, circle}, ->]
							\node{2}
								child{ node{1} }
								child{ node{3} };
						\end{tikzpicture}
					\end{tabular}
				\end{center}
			\item \textbf{parent} \\
				The opposite of a child. If a node \textit{A} is a child to node
				\textit{B}, then node \textit{B} is node \textit{A}'s \textbf{parent}.
			\item \textbf{ancestor} \\
				A node that can be reached when traversing a node's parents again and
				again.
			\item \textbf{descendant (subchild)} \\
				A node that can be reached when traversing a node's children.
			\item \textbf{leaf (external node)} \\
				A node with no children.
			\item \textbf{internal node} \\
				A node with at least 1 child.
			\item \textbf{degree} \\
				The number of children a node has.
			\item \textbf{edge} \\
				The path between two nodes.
			\item \textbf{root} \\
				Extending the tree analogy is the term \textbf{root}. The \textbf{root}
				is the root node is the top node in a tree. In the following diagram,
				3 is the root node.
				\begin{center}
					\begin{tikzpicture}[nodes={draw, circle}, ->]
						\node{3}
							child{ node{1} }
							child{ node{5}
								child{ node{4} }
								child{ node{6} } };
					\end{tikzpicture}
				\end{center}
			\item \textbf{level} \\
				1 + the number of edges between this node and the root. In other words
				the root is at level 1, its children are at level 2, those nodes'
				children are at level 3, and etc.
			\item \textbf{depth} \\
				The number of edges between this node and the root; level - 1.
			\item \textbf{height} \\
				The \textbf{height} of a node is the number of edges on the longest path
				between this node and a leaf. The \textbf{height} of a tree is the
				\textbf{height} of the root node.
			\item \textbf{subtree} \\
				A tree whose root belongs to another tree. For example in this tree
				\begin{center}
					\begin{tikzpicture}[nodes={draw, circle}, ->]
						\node{3}
							child{ node{1} }
							child{ node{5}
								child{ node{4} }
								child{ node{6} } };
					\end{tikzpicture}
				\end{center}
				You could say there is a subtree rooted at node 5, which is
				\begin{center}
					\begin{tikzpicture}[nodes={draw, circle}, ->]
							\node{5}
								child{ node{4} }
								child{ node{6} };
					\end{tikzpicture}
				\end{center}
			\item \textbf{full binary tree} \\
				A \textbf{full binary tree} is a tree where every node has two or zero
				children. For example the tree shown under subtree.
			\item \textbf{complete binary tree} \\
				A \textbf{complete binary tree} is a tree where every level is filled
				except for the last level (potentially). On the last level, the nodes
				are all filled in starting from the left.
				\begin{center}
					\begin{tikzpicture}[nodes={draw, circle}, ->]
						\node{4}
							child{ node{2}
								child{ node{1} }
								child{ node{3} } }
							child{ node{5} };
					\end{tikzpicture}
				\end{center}
			\item \textbf{perfect binary tree} \\
				A \textbf{perfect binary tree} is a tree where all nodes have two
				children and all leaves are at the same level.
				\begin{center}
					\begin{tikzpicture}[nodes={draw, circle}, ->]
						\node{4}
							child{ node{2} }
							child{ node{5} };
					\end{tikzpicture}
				\end{center}
		\end{itemize}

	\section{Description}
		One important feature of \textbf{binary search trees} is that all the nodes
		in the left subtree of a node \textit{A} will have values less than the value
		at node \textit{A} while the nodes of the right subtree will have greater
		values. For example, this is a valid \textbf{binary search tree}:
		\begin{center}
			\begin{tikzpicture}[nodes={draw, circle}, ->]
				\node{3}
					child{ node{1} }
					child{ node{5}
						child{ node{4} }
						child{ node{6} } };
			\end{tikzpicture}
		\end{center}
		However, if we change some of the numbers it may not be valid anymore (in this
		case it is not valid):
		\begin{center}
			\begin{tikzpicture}[nodes={draw, circle}, ->]
				\node{3}
					child{ node{1} }
					child{ node{5}
						child{ node{2} }
						child{ node{6} } };
			\end{tikzpicture}
		\end{center}
		Notice by changing 4 to 2, 2 is in 3's right subtree but is less than 3. Even
		though the parent-child relationship is still satisfied; namely that the left
		child is less than the parent ($2 < 5$), this is an \underline{invalid}
		\textbf{binary search tree}. This is a common mistake I see some students make.

		\noindent Typically, there are no duplicate nodes in a \textbf{binary search
		tree} so the relationship is strictly less or greater, however, if duplicates
		are allowed I'd guess it's ok to put them in either the left or right subtrees
		as long as it is consistent.

		\noindent Also, a \textbf{binary tree} and a \textbf{binary search tree} are
		not the same thing. Only a \textbf{binary search tree} has to exhibit this
		behaviour.

	\section{Searching}
		Why is this a constraint on the structure of the \textbf{binary search tree}?
		It's to make searching easier. Consider this slightly larger tree:
		\begin{center}
			\begin{tikzpicture}[nodes={draw, circle}, ->,
				level 1/.style={sibling distance=30mm},
				level 2/.style={sibling distance=15mm}]
				\node{10}
					child{ node{5}
						child{ node{3} }
						child{ node{7}
							child{ node{6} }
							child{ node{9} } } }
					child{ node{15}
						child{ node{11} }
						child{ node{99} } };
			\end{tikzpicture}
		\end{center}
		With only 3 comparisons, I can tell whether or not 14 is in the tree. Start
		at the root and ask if 14 is less or greater than 10. It's greater so it must
		be in the right subtree. Ask yourself the same thing again with the root of
		that subtree; is 14 greater or less than 15? It's less so it must be in the
		left subtree. Again, ask yourself the same questions with the new subtree's root
		and you'll answer that 14 is greater than 11, but 11 is a leaf node so 14 must
		not be in the tree. Thanks to the relationships between a node and its subtrees,
		we can quickly find a value if it exists.

		\noindent How can we find the maximum in the tree? Just keep heading right!
		\noindent What about the minimum? Just go left!

		\noindent However, there can be cases where your tree is not very useful for
		searching.
		Consider something like this:
		\begin{center}
			\begin{tikzpicture}[nodes={draw, circle}, ->]
				\node{10}
					child{ node{9}
						child{ node{8}
							child{ node {7} }
							child[ missing ] }
						child[ missing ] }
					child[ missing ];
			\end{tikzpicture}
		\end{center}
		This is looking like a normal list... Depending on what value we are looking
		for, we may have to traverse the entire tree before finding whether or not it
		exists. This kind of tree is known as a \textbf{degenerate} tree. It's not
		very important to know right now.

	\pagebreak

	\section{Insertion}
		Due to the magic of a \textbf{binary search tree}, insertion is very simple.
		First, search for the value you want to insert and you should eventually end
		up at a leaf node. Then attach a node to that leaf that follows the rules.
		In the searching example's tree, let's say we wanted to insert 14. First we
		searched for it and go to 11. Now we just tack 14 onto 11 and that's that.
		\begin{center}
			\begin{tikzpicture}[nodes={draw, circle}, ->,
				level 1/.style={sibling distance=30mm},
				level 2/.style={sibling distance=15mm}]
				\node{10}
					child{ node{5}
						child{ node{3} }
						child{ node{7}
							child{ node{6} }
							child{ node{9} } } }
					child{ node{15}
						child{ node{11}
							child[ missing ]
							child{ node{14} } }
						child{ node{99} } };
			\end{tikzpicture}
		\end{center}
	
	\section{Deletion}
		Deletion is slightly more complicated. There are 3 cases you want to consider:
		\begin{itemize}
			\item \textbf{no children} \\
				Want to remove a node with no children? Just remove it! For example
				remove 14 from the previous tree and you just get rid of it.
			\item \textbf{one child}
				Want to remove a node with 1 child? Just remove it (and link the child
				with the node's parent)! For example, removing 11 in the previous
				example will end up looking like
			\begin{center}
				\begin{tikzpicture}[nodes={draw, circle}, ->,
					level 1/.style={sibling distance=30mm},
					level 2/.style={sibling distance=15mm}]
					\node{10}
						child{ node{5}
							child{ node{3} }
							child{ node{7}
								child{ node{6} }
								child{ node{9} } } }
						child{ node{15}
							child{ node{14} }
							child{ node{99} } };
				\end{tikzpicture}
			\end{center}
			\item \textbf{two children}
				This is probably the most difficult case. Let's try removing the root
				of the above tree. What value should we replace it with? We want a value
				that is less than all the nodes in the right subtree and greater than
				all the nodes in the left subtree. So what about the minimum in the
				right subtree or the maximum in the left subtree? They both work
				and the one you take depends on the \textbf{strategy} you use. One is
				called the \textbf{predecessor replacement strategy} and the other is
				called the \textbf{successor replacement strategy}. To remember which
				one is which, consider the numbers in the tree in an ordered list
				\[
					... 7 \rightarrow 9 \rightarrow 10 \rightarrow 14 \rightarrow ...
				\]
				We are removing 10 and we can replace it with 9 or 14. If we are using
				the \textbf{predecessor replacement strategy}, we replace it with its
				\textbf{predecessor} which is 9 and vice versa. So let's see what the
				resulting tree looks like when you use the \textbf{successor replacement
				strategy}.
				\begin{center}
					\begin{tikzpicture}[nodes={draw, circle}, ->,
						level 1/.style={sibling distance=30mm},
						level 2/.style={sibling distance=15mm}]
						\node{14}
							child{ node{5}
								child{ node{3} }
								child{ node{7}
									child{ node{6} }
									child{ node{9} } } }
							child{ node{15}
								child[ missing ]
								child{ node{99} } };
					\end{tikzpicture}
				\end{center}
				That's not to bad is it? But wait we're not done. We also just removed
				14 from that location so it's possible there will be more work (in this
				case, 14 was a leaf node so it was easy)! Consider
				this example and the \textbf{predecessor replacement strategy}
				\begin{center}
					\begin{tikzpicture}[nodes={draw, circle}, ->,
						level 1/.style={sibling distance=30mm},
						level 2/.style={sibling distance=15mm}]
						\node{14}
							child{ node{10}
								child{ node{9} }
								child{ node{12}
									child{ node{11} }
									child[ missing ] } }
							child{ node{15}
								child[ missing ]
								child{ node{99} } };
					\end{tikzpicture}
				\end{center}
				Ok so we are removing 14. Based on the strategy, we will replace it
				with 12, but what will happen to 11? Treat moving 12 as the deletion
				of 12 from where it is, so it's like delting a node with 1 child! We
				end up with something like this:
				\begin{center}
					\begin{tikzpicture}[nodes={draw, circle}, ->,
						level 1/.style={sibling distance=30mm},
						level 2/.style={sibling distance=15mm}]
						\node{12}
							child{ node{10}
								child{ node{9} }
								child{ node{11} } }
							child{ node{15}
								child[ missing ]
								child{ node{99} } };
					\end{tikzpicture}
				\end{center}
				But is it possible to have to deal with another case where there
				will be 2 children? No, because to find the minimum or maximum of
				a tree you must keep heading right or left. You stop once there is no
				left or right child so the node you are replacing at \underline{cannot}
				have that child.
		\end{itemize}

	\section{Examples}
		Removed for now.

\end{document}
