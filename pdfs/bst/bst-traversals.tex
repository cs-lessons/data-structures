\documentclass[11pt]{book}
\usepackage[margin=0.5in]{geometry}
\usepackage{amsmath}
\usepackage{tikz}
\usetikzlibrary{graphdrawing.trees}

\begin{document}
	\setcounter{chapter}{3}
	\setcounter{section}{7}
	\section{Traversals/Searches}
		There are various traversals that you can do on a binary tree.
	\subsection{Breadth First Traversal/Search (BFS)}
		The idea of a BFS is to travel along each level of the tree in order
		starting from its root. Consider the following tree: \\
		\begin{center}
			\begin{tikzpicture}[nodes={draw, circle}, ->]
				\node{3}
					child{ node{1} }
					child{ node{5}
						child{ node{4} }
						child{ node{6} } };
			\end{tikzpicture}
		\end{center}
		What we want to do is traverse this tree in an order such that the output
		will be $3, 1, 5, 4, 6$. One way to do this is to rely on a queue and follow
		the algorithm:
		\begin{verbatim}
		Q <- new Queue
		enqueue root
		while Q is not empty:
		    node <- dequeue from Q
		    enqueue node.left
		    enqueue node.right
		    print node
		\end{verbatim}
		Let's try this on the above tree. I'll just list the queue contents and what is
		printed after each iteration of the loop.
		\begin{align}
			& Q = [3] & \text{    } & P = [] \\
			& Q = [1,5] & \text{    } & P = [3] \\
			& Q = [5] & \text{    } & P = [3,1] \\
			& Q = [4,6] & \text{    } & P = [3,1,5] \\
			& Q = [6] & \text{    } & P = [3,1,5,4] \\
			& Q = [] & \text{    } & P = [3,1,5,4,6] \\
		\end{align}

	\subsection{Depth First Traversal/Search (DFS)}
		The idea of a DFS is to go as deep as possible into the tree as possible
		while searching. One way to achieve this is to replace the queue in the
		previous algorithm with a stack and enqueue/dequeue with push/pop respectively.
		Let's see how that goes on the same tree.
		\begin{align}
			& S = [3] & \text{    } & P = [] \\
			& S = [1,5] & \text{    } & P = [3] \\
			& S = [1,4,6] & \text{    } & P = [3,5] \\
			& S = [1,4] & \text{    } & P = [3,5,6] \\
			& S = [1] & \text{    } & P = [3,5,6,4] \\
			& S = [] & \text{    } & P = [3,5,6,4,1] \\
		\end{align}
		An alternate way to perform a DFS is to take advantage of recursion to
		emulate the stack. It's as simple as
		\begin{verbatim}
		function dfs(node)
		    dfs(node.right)
		    dfs(node.left)
		    print node
		\end{verbatim}
		Notice that the output is actually different, however the idea was still the
		same; travel as deep into the tree as possible before doing anything.

	\subsection{Pre/In/Postorder Traversals}
		These traversals are pretty simple to remember if you memorize the format
		\begin{verbatim}
		function traverse(node)
		    // pre
		    traverse(node.left)
		    // in
		    traverse(node.right)
		    // post
		\end{verbatim}
		A preordering of a tree processes nodes in topological order. This means
		that the parent is processed before the children. This suggests that you
		should print the node before visiting the left and right children. \\

		\noindent An inorder traversal actually gives the contents of the binary search
		tree well, in order. This is done by printing the node after visiting the left
		subtree but before visiting the right, or as I prefer to say: in between the
		recursive calls. \\

		\noindent A postorder traversal is the final kind of tree traversal and can be
		executed by printing the node after traversing both left and right subtrees. \\

		\noindent If we were to perform these three traversals on the above tree, we
		would get
		\begin{align}
			preorder & = & [3,1,5,4,6] \\
			inorder & = & [1,3,4,5,6] \\
			postorder & = & [1,4,6,5,3]
		\end{align}

		\noindent However, there is a simpler way to perform these traversals without
		stepping through the algorithm by hand. Let's look back at our tree:
		\begin{center}
			\begin{tikzpicture}[nodes={draw, circle}, ->]
				\node{3}
					child{ node{1} }
					child{ node{5}
						child{ node{4} }
						child{ node{6} } };
			\end{tikzpicture}
		\end{center}
		We notice that our left right recursive calls more or less traverse the
		tree in a fashion that outlines the structure of the tree starting from
		the left of the root; $[3,1,5,4,6]$. All we have to do to mimic when we
		print the node is to mark the node at a spot that represents when it is
		printed i.e. these 3 spots.
		\begin{center}
			\begin{tikzpicture}[nodes={draw, circle}, ->]
				\node[draw] at (1, 1) (n){n};
				\node[draw=none] at (0.3, 1) (l);
				\node[draw=none] at (1, 0.3) (b);
				\node[draw=none] at (1.7, 1) (r);

				\draw[->] (b) -- (n);
				\draw[->] (l) -- (n);
				\draw[->] (r) -- (n)
			\end{tikzpicture}
		\end{center}
		I'll leave it to you to figure out which spot is for which traversal.
\end{document}
