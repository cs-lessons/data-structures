\documentclass[11pt]{book}
\usepackage[margin=0.5in]{geometry}
\usepackage{makecell}

\begin{document}
	\setcounter{chapter}{1}
	\chapter{Stacks}
	\section{Introduction}
		The \textbf{stack} is another simple yet pervasive data structure. You
		may have heard the word \textbf{stack} used to refer to a pile of things,
		for example, a stack of books. This comparison can be used to easily
		understand the functionality of a \textbf{stack}.

		Also, \textbf{stack} may also refer to the set of technologies that are
		being used for a project. Make sure not to confuse the two, we are
		talking about data structures here.

	\section{Description}
		\textbf{Stacks} work in a LIFO (last-in first-out) manner unlike the
		\textbf{queue}. This means the last thing to enter the stack will be
		the primary target for removal. Let's bring back the stack of books
		comparison back; when adding to the stack of books, you put something
		on the top, and when removing something from the stack, you also take
		from the top.

	\section{Implementation}
		\begin{itemize}
			\item \textbf{push} \\
				\textbf{Push} is used to add something to the top of the stack
			\item \textbf{pop} \\
				\textbf{Pop} is used to remove from the top of the stack
		\end{itemize}
		
	\section{Examples}
		\begin{enumerate}
			\item Push \\
				\textbf{Push} 9 onto the following stack (arrows point towards
				the top).
				\[
					8 \rightarrow 1 \rightarrow 0
				\]
				\[
					Ans: 9 \rightarrow 8 \rightarrow 1 \rightarrow 0
				\]
			\item Pop \\
				What do you get when you \textbf{pop} from the solution to the
				previous question?
				\[
					Ans: 9
				\]
		\end{enumerate}

\end{document}
