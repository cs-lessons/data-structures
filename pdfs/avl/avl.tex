\documentclass[11pt]{book}
\usepackage[margin=0.5in]{geometry}
\usepackage{makecell}
\usepackage{tikz}
\usetikzlibrary{graphdrawing.trees}

\begin{document}
	\setcounter{chapter}{4}
	\chapter{AVL Trees}
	\section{Introduction}
		Remember those few minutes weeks ago where we talked about degenerate
		binary trees and their bad qualities? Well, what if we had a way to avoid
		that... We do! With \textbf{AVL trees}! They do fancy things to prevent this.
		They're called \textbf{AVL trees} because they're named after someone but that's
		not important for knowing how they work. Sorry inventors! \\

		\noindent Do note that \textbf{AVL trees} are only one of several version of
		self-balancing binary trees. Another example would be a red-black tree
		which may or may not be covered depending on how much time there is left
		at the end of the semester.

	\section{Description}
		You've heard of binary search trees, well the \textbf{AVL tree} is a binary
		search tree but a lot fancier. The important thing here is \textbf{balance}.
		When a tree is balanced, this means the heights of the two child subtrees of
		a node can only differ by up to 1. Let's take a look at multiple examples to
		understand this. \\

		\noindent Consider this tree. \\

		\noindent What tree? The empty tree of course! Naturally it falls under the
		category of an \textbf{AVL tree}. Jokes aside, let's look at a tree with 1 node.

		\begin{center}
			\begin{tikzpicture}[nodes={draw, circle}, ->]
				\node{10}
			\end{tikzpicture}
		\end{center}

		\noindent Consider the one node. Do its two child subtrees' heights differ by more
		than 1? No, both of those subtrees are empty and have the same height. For this class
		empty trees are considered to have a height of -1. Alright let's add another node.

		\begin{center}
			\begin{tikzpicture}[nodes={draw, circle}, ->]
				\node{10}
					child{ node{5} }
					child[ missing ];
			\end{tikzpicture}
		\end{center}

		\noindent Is this an \textbf{AVL tree}? Alright, heights for child subtrees for 10
		are 0 and -1 which works, then heights for child subtrees for 5 are -1 and -1 which
		also works, so it's good. Let's get a little more complicated:

		\begin{center}
			\begin{tikzpicture}[nodes={draw, circle}, ->]
				\node{10}
					child{ node{5} }
					child{ node{15}
						child{ node{13} }
						child[ missing ] };
			\end{tikzpicture}
		\end{center}

		\noindent 10 has heights 0 and 1, 5 has heights -1 and -1, 15 has heights 0 and -1,
		13 has heights -1 and -1, so yes again this is an \textbf{AVL tree}. Now for a large
		example:

		\begin{center}
			\begin{tikzpicture}[nodes={draw, circle}, ->,
				level 1/.style={sibling distance=30mm},
				level 2/.style={sibling distance=15mm}]
				\node{10}
					child{ node{5}
						child{ node{2} }
						child[ missing ] }
					child{ node{15}
						child{ node{13} }
						child{ node{18}
							child{ node{17} }
							child[ missing ] } };
			\end{tikzpicture}
		\end{center}

		\noindent Is this an \textbf{AVL tree}? Well I'll tell you it is, but you will have
		to confirm it. Actually there's something interesting going on here that I'll let you
		in on.

	\section{Searching}
		This is a binary search tree, searching works no differently.

	\pagebreak

	\section{Self Balancing}
		Whenever the contents of an \textbf{AVL tree} are modified, there is a
		potential for rebalancing the tree. One way to keep track of the balance
		is to keep a number known as the \textbf{balance factor} at each node.
		This number keeps track of the difference of the heights of the left and
		right subtrees. Let's look at one of the previous trees:

		\begin{center}
			\begin{tikzpicture}[nodes={draw, circle}, ->,
				level 1/.style={sibling distance=30mm},
				level 2/.style={sibling distance=15mm}]
				\node{10}
					child{ node{5}
						child{ node{2} }
						child[ missing ] }
					child{ node{15}
						child{ node{13} }
						child{ node{18}
							child{ node{17} }
							child[ missing ] } };
			\end{tikzpicture}
		\end{center}

		\noindent Right away, we can go ahead and assign the leaves a \textbf{balance factor}
		of 0 because their subtrees' heights differ by 0. What about for the node 15? The
		heights of its left and right subtrees are 0 and 1 respectively. We can subtract the
		left tree's height from the right tree's to get a \textbf{balance factor} of -1. We
		do this for the rest of the nodes and end up with something like this:

		\begin{center}
			\begin{tikzpicture}[nodes={draw, circle}, ->,
				level 1/.style={sibling distance=30mm},
				level 2/.style={sibling distance=15mm}]
				\node[label=-1]{10}
					child{ node[label=1]{5}
						child{ node[label=0]{2} }
						child[ missing ] }
					child{ node[label=-1]{15}
						child{ node[label=0]{13} }
						child{ node[label=1]{18}
							child{ node[label=0]{17} }
							child[ missing ] } };
			\end{tikzpicture}
		\end{center}

		\noindent Whats important to note is that all of these balance factors are between
		-1 and 1 (inclusive). If they were any higher or lower, then that would mean the
		heights differ by more than 1 and a \textbf{rotation} would be needed.

	\pagebreak

	\section{Insertion}
		Insertion is like a normal binary search tree. Let's take out previous tree and
		insert 16.

		\begin{center}
			\begin{tikzpicture}[nodes={draw, circle}, ->,
				level 1/.style={sibling distance=30mm},
				level 2/.style={sibling distance=15mm}]
				\node{10}
					child{ node{5}
						child{ node{2} }
						child[ missing ] }
					child{ node{15}
						child{ node{13} }
						child{ node{18}
							child{ node{17}
								child{ node{16} }
								child[ missing ] }
							child[ missing ] } };
			\end{tikzpicture}
		\end{center}

		\noindent Hopefully you have the hang of inserting into a normal binary search tree.
		But wait, let's also update the balance factors.

		\begin{center}
			\begin{tikzpicture}[nodes={draw, circle}, ->,
				level 1/.style={sibling distance=30mm},
				level 2/.style={sibling distance=15mm}]
				\node[label=-2]{10}
					child{ node[label=1]{5}
						child{ node[label=0]{2} }
						child[ missing ] }
					child{ node[label=-2]{15}
						child{ node[label=0]{13} }
						child{ node[label=2]{18}
							child{ node[label=-1]{17}
								child{ node[label=0]{16} }
								child[ missing ] }
							child[ missing ] } };
			\end{tikzpicture}
		\end{center}

		\noindent Uh oh, we now have (-)2s as some balance factors. This means we need to
		\textbf{rotate}. We start from the node we inserted and work our way towards the
		root until we find a balance factor that is too high or too low. This is node 18.
		We then pick its children whose balance factors are consecutive to it so we will
		pick these 3 nodes:

		\begin{center}
			\begin{tikzpicture}[nodes={draw, circle}, ->]
						\node[label=-2]{18}
							child{ node[label=-1]{17}
								child{ node[label=0]{16} }
								child[ missing ] }
							child[ missing ];
			\end{tikzpicture}
		\end{center}

		\noindent, now we want to balance this so what we will rotate these nodes to lower
		the overall height yet keep them as a valid binary search tree arrangement. Can
		you see how this is done? Just like this:

		\begin{center}
			\begin{tabular*}{0.75\textwidth}{@{\extracolsep{\fill}}cccc}
				&
				\begin{tikzpicture}[nodes={draw, circle}, ->]
							\node{18}
								child{ node{17}
									child{ node{16} }
									child[ missing ] }
								child[ missing ];
				\end{tikzpicture} &
				\begin{tikzpicture}[nodes={draw, circle}, ->]
							\node{17}
								child{ node{16} }
								child{ node{18} };
				\end{tikzpicture}
				&
			\end{tabular}
		\end{center}

		\noindent And all that's left is to put it back where it was!  As you can see, the
		new balance factors are good.

		\begin{center}
			\begin{tikzpicture}[nodes={draw, circle}, ->,
				level 1/.style={sibling distance=30mm},
				level 2/.style={sibling distance=15mm}]
				\node[label=-1]{10}
					child{ node[label=1]{5}
						child{ node[label=0]{2} }
						child[ missing ] }
					child{ node[label=-1]{15}
						child{ node[label=0]{13} }
						child{ node[label=0]{17}
								child{ node[label=0]{16} }
								child{ node[label=0]{18} } } };
			\end{tikzpicture}
		\end{center}

		\noindent There are more ways to rotate, but I have a simpler way to remember
		them all which will be detailed later.

	\section{Deletion}
		Deletion is the same as a binary search tree, just make sure to rotate if necessary
		after.

	\pagebreak

	\section{Easy Rotations}
		Rotations are the core idea behind \textbf{AVL trees} and may be tough. Let's make
		try to make them simpler. \\

		\noindent One observation you should be able to make is that when inserting or
		deleting from a binary search tree, only the heights of the subtrees rooted at nodes
		on the path from the inserted or deleted node to the root are affected. This also
		suggests that only balance factors along this path are changed .It's a mouthful but
		let's look at what it means. By going through our insertion example.

		\begin{center}
			\begin{tabular*}{0.75\textwidth}{@{\extracolsep{\fill}}cccc}
				&
				\begin{center}
					\begin{tikzpicture}[nodes={draw, circle}, ->,
						level 1/.style={sibling distance=30mm},
						level 2/.style={sibling distance=15mm}]
						\node[label=-1]{10}
							child{ node[label=1]{5}
								child{ node[label=0]{2} }
								child[ missing ] }
							child{ node[label=-1]{15}
								child{ node[label=0]{13} }
								child{ node[label=1]{18}
									child{ node[label=0]{17} }
									child[ missing ] } };
					\end{tikzpicture}
				\end{center} &
				\begin{tikzpicture}[nodes={draw, circle}, ->,
					level 1/.style={sibling distance=30mm},
					level 2/.style={sibling distance=15mm}]
					\node[label=-2]{10}
						child{ node[label=1]{5}
							child{ node[label=0]{2} }
							child[ missing ] }
						child{ node[label=-2]{15}
							child{ node[label=0]{13} }
							child{ node[label=2]{18}
								child{ node[label=-1]{17}
									child{ node[label=0]{16} }
									child[ missing ] }
								child[ missing ] } };
				\end{tikzpicture} &
				&
			\end{tabular}
		\end{center}

	\noindent Remember we inserted 16. The balance factors that changed are the ones on the
	nodes from 16 to the root, 10. This shows us where we may have to rebalance. \\

	\noindent Picking out 3 nodes is pretty simple - just work your way up from 16. But once
	you find them, how should you rotate them? How many ways can you arrange 3 nodes such
	that they are unbalanced? Exactly 4, and since you know that you need to keep the nodes
	arranged so they are a valid binary search tree, there's no reason to memorize anything
	(although I've pictured them below). \\

	\noindent One important thing to note is that there may be other subtrees connected
	to these 3 nodes. Similarly for these, you can also reason out there new positions if
	necessary.

	\pagebreak

	\begin{center}
		\begin{tabular*}{0.75\textwidth}{@{\extracolsep{\fill}}cccc}
			&
			\begin{tikzpicture}[nodes={draw, circle}, ->]
						\node{a}
							child{ node{b}
								child{ node{c} }
								child[ missing ] }
							child[ missing ];
			\end{tikzpicture} &
			\begin{tikzpicture}[nodes={draw, circle}, ->]
						\node{b}
							child{ node{c} }
							child{ node{a} };
			\end{tikzpicture} &
			&
		\end{tabular}
		\begin{tabular*}{0.75\textwidth}{@{\extracolsep{\fill}}cccc}
			&
			\begin{tikzpicture}[nodes={draw, circle}, ->]
						\node{a}
							child[ missing ]
							child{ node{b}
								child[ missing ]
								child{ node{c} } };
			\end{tikzpicture} &
			\begin{tikzpicture}[nodes={draw, circle}, ->]
						\node{b}
							child{ node{a} }
							child{ node{c} };
			\end{tikzpicture} &
			&
		\end{tabular}
		\begin{tabular*}{0.75\textwidth}{@{\extracolsep{\fill}}cccc}
			&
			\begin{tikzpicture}[nodes={draw, circle}, ->]
						\node{a}
							child[ missing ]
							child{ node{b}
								child{ node{c} }
								child[ missing ] };
			\end{tikzpicture} &
			\begin{tikzpicture}[nodes={draw, circle}, ->]
						\node{c}
							child{ node{a} }
							child{ node{b} };
			\end{tikzpicture} &
			&
		\end{tabular}
		\begin{tabular*}{0.75\textwidth}{@{\extracolsep{\fill}}cccc}
			&
			\begin{tikzpicture}[nodes={draw, circle}, ->]
						\node{a}
							child{ node{b}
								child[ missing ]
								child{ node{c} } }
							child[ missing ];
			\end{tikzpicture} &
			\begin{tikzpicture}[nodes={draw, circle}, ->]
						\node{c}
							child{ node{b} }
							child{ node{a} };
			\end{tikzpicture} &
			&
		\end{tabular}
	\end{center}

	\noindent One final note - you may need to rotate multiple times after an insertion or
	deletion.

	\section{Examples}
		Removed for now.

\end{document}
