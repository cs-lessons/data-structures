\documentclass[11pt]{book}
\usepackage[margin=0.5in]{geometry}
\usepackage{makecell}

\begin{document}
	\chapter{Queues}
	\section{Introduction}
		\textbf{Queues} are one of the simplest data structures that one will
		learn about. Most likely you have heard the word \textbf{queue} in the
		context of a waiting line or such. The idea of a \textbf{queue} in
		Computer Science is really no different.

	\section{Description}
		\textbf{Queues} work in a FIFO (first-in first-out) manner. This simply
		means that the first thing into the \textbf{queue} will be the first
		thing out, or if extended, items that enter the \textbf{queue} earlier
		exit the \textbf{queue} earlier. To build on the waiting line example,
		someone who lines up earlier than you will reach the end of the
		\textbf{queue} and exit earlier than you.

	\section{Implementation}
		To support this behaviour, \textbf{queues} implement the following
		methods:\footnote{Depending on your programming language, the method
		names may not be accurate. Nonetheless, there should be methods that
		provide identical functionality}
		\begin{itemize}
			\item \textbf{enqueue} \\
				\textbf{Enqueue} is used to add elements to the back of the
				\textbf{queue}
			\item \textbf{dequeue} \\
				\textbf{Dequeue} is used to remove elements from the front of
				the \textbf{queue}
		\end{itemize}
		
	\section{Examples}
		\begin{enumerate}
			\item Enqueueing \\
				Starting with the following \textbf{queue} (arrows point towards
				the front), \textbf{enqueue} 3.
				\[
					8 \rightarrow 9 \rightarrow 0
				\]
				\[
					Ans: 3 \rightarrow 8 \rightarrow 9 \rightarrow 0
				\]
			\item Dequeueing \\
				What do you get when you \textbf{dequeue} from the previous
				solution?
				\[
					Ans: 0
				\]
		\end{enumerate}
\end{document}
