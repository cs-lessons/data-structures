\documentclass[11pt]{book}
\usepackage[margin=0.5in]{geometry}
\usepackage{makecell}
\usepackage{tikz}
\usetikzlibrary{graphdrawing.trees}

\begin{document}
	\setcounter{chapter}{6}
	\chapter{Graphs}
	\section{Introduction}
		You may have heard about graphs before, however, graphs in CS are not
		like graphs you have previously encountered in Mathematics... unless
		you've taken graph theory or something.

	\section{Terminology}
		\begin{itemize}
			\item \textbf{vertex} \\
				A vertex is like a node for a tree. Actually, you might hear me
				say node instead of vertex a lot because they're interchangeable
				in this case. They will be represented similarly. \\
				\begin{center}
					\begin{tikzpicture}[nodes={draw, circle}, ->]
						\node{a};
					\end{tikzpicture}
				\end{center}
			\item \textbf{edge} \\
				An edge is the line between two vertices in a graph. It shows the
				relationship between the two vertices. Here is an edge with length 1
				between vertices a and b. \\
				\begin{center}
					\begin{tikzpicture}[nodes={draw, circle}, ->]
						\tikzset{edge/.style={midway,above,draw=none}}
						\node[draw] at (0, 0) (a){a};
						\node[draw] at (2, 0) (b){b};

						\draw[-] (a) -- (b) node[edge]{1};
					\end{tikzpicture}
				\end{center}
				Edges can be directed or undirected, which is explained lower. \\
			\item \textbf{weight} \\
				The weight of an edge is the number assigned to the edge. \\
			\item \textbf{adjacent} \\
				Vertices a and b are adjacent to each other if there is a edge
				between a and b. \\
			\item \textbf{loop} \\
				A loop is an edge that extends from a vertex to itself. \\
				\begin{center}
					\begin{tikzpicture}[nodes={draw, circle}, ->]
						\tikzset{edge/.style={midway,above,draw=none}}
						\node[draw] at (0, 0) (a){a};

						\draw[->] (a) to [out=110,in=70,looseness=8] (a);
					\end{tikzpicture}
				\end{center}


			\pagebreak

			\item \textbf{undirected graph} \\
				An undirected graph is a graph whose edges have no direction. \\
				\begin{center}
					\begin{tikzpicture}[nodes={draw, circle}, ->]
						\tikzset{edge/.style={midway,above,draw=none}}
						\node[draw] at (0, 0) (a){a};
						\node[draw] at (2, 0) (b){b};
						\node[draw] at (0, 2) (c){c};
						\node[draw] at (2, 2) (d){d};

						\draw[-] (a) -- (c) node[edge,left]{2};
						\draw[-] (a) -- (b) node[edge]{2};
						\draw[-] (b) -- (c) node[edge,right]{$2\sqrt{2}$};
						\draw[-] (c) -- (d) node[edge]{2};
					\end{tikzpicture}
				\end{center}
				As you can see, the edge $\{a, c\}$ is the same as the edge $\{c, a\}$ and
				the same is true for the other edges.
			\item \textbf{directed graph} \\
				A directed graph is a graph whose edges have direction. \\
				\begin{center}
					\begin{tikzpicture}[nodes={draw, circle}, ->]
						\tikzset{edge/.style={midway,above,draw=none}}
						\node[draw] at (0, 0) (a){a};
						\node[draw] at (2, 0) (b){b};
						\node[draw] at (0, 2) (c){c};
						\node[draw] at (2, 2) (d){d};

						\draw[->] (a) -- (c) node[edge,left]{2};
						% the numbers in the positioning is the angle of the circle
						% to start from
						\draw[->] (a.10) -- (b.170) node[edge]{2};
						\draw[->] (b.190) -- (a.-10) node[edge,below]{1};
						\draw[->] (b) -- (c) node[edge,right]{$2\sqrt{2}$};
						\draw[->] (c) -- (d) node[edge]{2};
					\end{tikzpicture}
				\end{center}
				Here we can see that the edge $\{a, c\}$ exists but the edge $\{c, a\}$
				doesn't. Furthermore, we can see that the edges $\{a, b\}$ and $\{b, a\}$
				exist but have different weights.
			\item \textbf{mixed graph} \\
				A mixed graph is a graph that has both directed and undirected
				edges. In other words, you can consider a directed graph a mixed
				graph with 0 undirected edges and vice versa for an undirected graph. \\
			\item \textbf{simple graph} \\
				A simple graph is a graph that does not have multiple edges between
				any pair of vertices and does not have any loops.
			\item \textbf{weighted graph} \\
				A weighted graph is a graph whose edges have weights. \\
			\item \textbf{complete graph} \\
				A complete graph is a graph where all pairs of vertices have an edge.
				In other words, all possible edges exist.
				\begin{center}
					\begin{tikzpicture}[nodes={draw, circle}, ->]
						\tikzset{edge/.style={midway,above,draw=none}}
						\node[draw] at (0, 0) (a){a};
						\node[draw] at (2, 0) (b){b};
						\node[draw] at (0, 2) (c){c};
						\node[draw] at (2, 2) (d){d};

						\draw[-] (a) -- (b);
						\draw[-] (a) -- (c);
						\draw[-] (a) -- (d);
						\draw[-] (b) -- (c);
						\draw[-] (b) -- (d);
						\draw[-] (c) -- (d);
					\end{tikzpicture}
				\end{center}

			\pagebreak

			\item \textbf{connected graph} \\
				A connected graph is a graph where there is a path between all pairs
				of vertices, otherwise it is a disconnected graph. Here is an example
				of a disconnected graph. \\
				\begin{center}
					\begin{tikzpicture}[nodes={draw, circle}, ->]
						\tikzset{edge/.style={midway,above,draw=none}}
						\node[draw] at (0, 0) (a){a};
						\node[draw] at (2, 0) (b){b};
						\node[draw] at (4, 0) (c){c};

						\draw[-] (a) -- (b);
					\end{tikzpicture}
				\end{center}

			\item \textbf{tree} \\
				A tree is a graph with no cycles. I'll leave that to you to confirm.
		\end{itemize}

	\section{Examples}
		Removed for now.

\end{document}
