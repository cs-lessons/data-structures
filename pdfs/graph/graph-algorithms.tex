\documentclass[11pt]{book}
\usepackage[margin=0.5in]{geometry}
\usepackage{amsmath}
\usepackage{tikz}
\usetikzlibrary{graphdrawing.trees}

\begin{document}
	\setcounter{chapter}{6}
	\setcounter{section}{3}
	\section{Graph Algorithms}
		Similar to trees, you can do the same traversals.
	\subsection{Breadth First Traversal/Search (BFS)}
		\begin{center}
			\begin{tikzpicture}[nodes={draw, circle}, ->]
				\tikzset{edge/.style={midway,above,draw=none}}
				\node[draw] at (0, 0) (a){a};
				\node[draw] at (2, 0) (b){b};
				\node[draw] at (0, 2) (c){c};
				\node[draw] at (2, 2) (d){d};
				\node[draw] at (3, 1) (e){e};
				\node[draw] at (1, 3) (f){f};

				\draw[-] (a) -- (b);
				\draw[-] (a) -- (c);
				\draw[-] (a) -- (d);
				\draw[-] (b) -- (c);
				\draw[-] (b) -- (d);
				\draw[-] (c) -- (d);
				\draw[-] (e) -- (d);
				\draw[-] (e) -- (b);
				\draw[-] (c) -- (f);
				\draw[-] (d) -- (f);
			\end{tikzpicture}
		\end{center}
		Consider this graph to be a map of where you are and let's say you start
		at vertex A. When performing a BFS, you look at everything directly
		adjacent to you first, like you're scanning the areas side to side
		(hence breadth). This means vertices B and C, then D E F. In other words,
		we visit verticies that are 1 edge away from the starting point, then
		we visit verticies that are 2 edges away from the starting point. How
		do we do this? \\

		\noindent Remember the BFS algorithm for trees? We can modify it slightly
		to work for graphs.
		\begin{verbatim}
		Q <- new Queue
		enqueue start
		while Q is not empty:
		    x <- dequeue from Q
		    for each adjacent unvisited vertex y to x
		        enqueue y
		        mark y as visited
		    print y
		\end{verbatim}
		All we did is replace root with the initial vertex and enqueue all
		adjacent unvisited vertices rather than left or right. We have to mark
		vertices as visited because graphs can have cycles and we do not want
		to visit the same location twice. Let's see the output of this algorithm
		on the example graph.
		\begin{align}
			& Q = [a] & \text{    } & P = [] \\
			& Q = [b,c] & \text{    } & P = [a] \\
			& Q = [c,e,d] & \text{    } & P = [a,b] \\
			& Q = [e,d,f] & \text{    } & P = [a,b,c] \\
			& Q = [d,f] & \text{    } & P = [a,b,c,e] \\
			& Q = [f] & \text{    } & P = [a,b,c,e,d] \\
			& Q = [] & \text{    } & P = [a,b,c,e,d,f]
		\end{align}
		And as we can see it works out. It's not a big deal if we visit E before
		or after D, however for the exams there will probably be some guidelines
		to ensure you get a consistent output like visiting adjacent nodes in a
		certain order.

	\pagebreak

	\subsection{Depth First Traversal/Search (DFS)}
		Again, similar to trees, we go as deep as we can into the graph. The
		important thing to note is that in this case deepness is not necessarily
		number of edges from the start node. Let's modify the graph to emphasize
		this point.
		\begin{center}
			\begin{tikzpicture}[nodes={draw, circle}, ->]
				\tikzset{edge/.style={midway,above,draw=none}}
				\node[draw] at (0, 0) (a){a};
				\node[draw] at (2, 0) (b){b};
				\node[draw] at (0, 2) (c){c};
				\node[draw] at (2, 2) (d){d};
				\node[draw] at (3, 1) (e){e};
				\node[draw] at (1, 3) (f){f};
				\node[draw] at (3, 3) (g){g};

				\draw[-] (a) -- (b);
				\draw[-] (a) -- (c);
				\draw[-] (a) -- (d);
				\draw[-] (b) -- (c);
				\draw[-] (b) -- (d);
				\draw[-] (c) -- (d);
				\draw[-] (e) -- (d);
				\draw[-] (e) -- (b);
				\draw[-] (c) -- (f);
				\draw[-] (d) -- (f);
				\draw[-] (g) -- (f);
			\end{tikzpicture}
		\end{center}
		Just like the BFS and DFS algorithms for trees, the DFS algorithm for
		graphs can be obtained by replacing the queue with a stack and its
		respective operations. Or recursion.
		\begin{align}
			& S = [a] & \text{    } & P = [] \\
			& S = [c,b] & \text{    } & P = [a] \\
			& S = [f,d,b] & \text{    } & P = [a,c] \\
			& S = [g,d,b] & \text{    } & P = [a,c,f] \\
			& S = [d,b] & \text{    } & P = [a,c,f,g] \\
			& S = [e,b] & \text{    } & P = [a,c,f,g,d] \\
			& S = [b] & \text{    } & P = [a,c,f,g,d,e] \\
			& S = [] & \text{    } & P = [a,c,f,g,d,e,b]
		\end{align}
		Notice how the last thing that is visited is actually 1 edge away from
		where we started. Furthermore, depending on how the contents are printed
		out, we can find some pretty interesting arrangements.

	\subsection{Toplogical Sort}
		To continue off of the previous idea we touched upon, we can get a
		topological ordering of a graph by utilizing a depth first traversal. \\

		\noindent What is a topological ordering? It's an ordering of vertices
		such that for every directed edge $\{u, v\}$ u comes before v. One example
		would be our course requirements. You guys couldn't take data structures
		before intro, and compilers before data structures. By performing a
		topological sort, we can figure out what the prerequisites for courses
		are, or a method of taking courses in an order so that prerequisites are
		fulilled before taking the actual course. \\

		\noindent For this, we do a post order DFS (do something e.g. print after
		visiting adjacent nodes) then REVERSE the output. \\

		\noindent Consider the following graph:
		\begin{center}
			\begin{tikzpicture}[nodes={draw, circle}, ->]
				\tikzset{edge/.style={midway,above,draw=none}}
				\node[draw] at (0, 0) (a){a};
				\node[draw] at (2, 0) (b){b};
				\node[draw] at (4, 0) (c){c};
				\node[draw] at (0, 2) (d){d};
				\node[draw] at (2, 2) (e){e};
				\node[draw] at (0, 4) (g){g};
				\node[draw] at (2, 4) (h){h};
				\node[draw] at (4, 4) (i){i};

				\draw[->] (d) -- (a);
				\draw[->] (g) -- (d);
				\draw[->] (d) -- (b);
				\draw[->] (d) -- (c);
				\draw[->] (e) -- (b);
				\draw[->] (h) -- (d);
				\draw[->] (h) -- (e);
				\draw[->] (i) -- (e);
				\draw[->] (i) -- (c);
			\end{tikzpicture}
		\end{center}
		First we're doing a postordering which means print after everything,
		so instead of popping/dequeueing, we just peek/front. Or we can go
		the simple recursive way.
		\begin{verbatim}
		function dfs_helper(x)
		    mark x as visited
		    for each adjacent unvisited vertex y to x
		        dfs_helper(y)
		    print x

		function topological_sort()
		    for each vertex with no outgoing edges x
		        if x is unvisited
		            dfs(x)
		\end{verbatim}
		After everything, we will print out the following: $[a, b, c, d, g, e, h,
		i]$. But remember, we wanted it reversed, so it would come out as $[i, h,
		e, g, d, c, b, a]$. One way we can do this is to push to a stack instead
		of a print and at the end of dfs, we pop and print from the stack until
		it is empty. As such, a quick observation is all that's needed to confirm
		that this is a proper topological ordering.
		\begin{itemize}
			\item d completed before a? check
			\item d completed before b? check
			\item d completed before c? check
			\item d completed before c? check
			\item g and h completed before d? check
			\item h and i completed before e? check
		\end{itemize}

	\subsection{Djikstra's Algorithm}
		Naturally when thinking about graphs and maps, you might want to find the
		shortest path between points. Djikstra's (check spelling) algorithm is
		one way to do this. Specifically, Djikstra's algorithm is able
		to determine the shortest path from one vertex to all other vertices as
		long as there are no negative weights. \\

		\noindent I find the simplest way to memorize this algorithm is to
		remember that it is greedy. How is it greedy? You always take the next
		shortest path possible that takes you to an unvisited vertex. Let's look
		at an example

		\begin{center}
			\begin{tikzpicture}[nodes={draw, circle}, ->]
				\tikzset{edge/.style={midway,above,draw=none}}
				\node[draw] at (0, 0) (a){a};
				\node[draw] at (2, 0) (b){b};
				\node[draw] at (4, 0) (c){c};
				\node[draw] at (0, 2) (d){d};
				\node[draw] at (2, 2) (e){e};
				\node[draw] at (4, 2) (f){f};
				\node[draw] at (0, 4) (g){g};
				\node[draw] at (2, 4) (h){h};
				\node[draw] at (4, 4) (i){i};

				\draw[->] (d) -- (a) node[edge,left]{5};
				\draw[->] (g) -- (d) node[edge, left]{3};
				\draw[->] (d) -- (b) node[edge, below]{10};
				\draw[->] (d) -- (c) node[edge, right]{12};
				\draw[->] (e) -- (b) node[edge, below]{20};
				\draw[->] (h) -- (d) node[edge, left]{1};
				\draw[->] (h) -- (e) node[edge, left]{4};
				\draw[->] (i) -- (e) node[edge, left]{3};
				\draw[-] (i) edge[bend left] node[edge, right]{10} (c);
				\draw[-] (f) -- (e) node[edge]{6};
				\draw[-] (e) -- (d) node[edge]{7};
				\draw[-] (i) -- (h) node[edge]{3};
				\draw[-] (g) -- (h) node[edge]{12};
			\end{tikzpicture}
		\end{center}
		Consider this beast of a graph (sorry I'm not good at positioning the
		weights). First, notice this is a mixed graph because it has both
		directed and undirected edges. With all our initial observations done,
		let's try to find the shortest path from $e$ to all other vertices. \\

		\noindent We start from $e$ and right away we know that the shortest path
		from $e$ to $e$ is 0. Now we look at all adjacent verticies for all
		vertices that we have visited so far. We visited $e$ so we can reach
		$f, d, b$ for 6, 7, and 20 respectively. The shortest path is 6, so
		we visit $f$ and now we know the shortest path to $f$ is 6. Now we do
		the same thing; we can visit $d, b$ for 7 and 20 so we go to $d$. Now
		we know the shortest path to $d$ is 7. And one more time before I show
		our current progress. Now that we have visited $e, f, d$ and know the
		shortest paths between $e$ and those vertices, we can now visit $b$ and
		now $a$ and $c$ by going through $d$. Furthermore, notice that we can
		also visit $b$ by going through $d$ which is actually a shorter path!
		Let's see a table that shows our current progress:

		\begin{center}
			\begin{tabular}{|c|c|c|c|c|}
				\hline
				\_ & \_ & e & f & d \\
				\hline
				a & & & & 12 \\
				\hline
				b & 20 & 20 & 20 & 17 \\
				\hline
				c & & & & 19 \\
				\hline
				d & 7 & 7 & \textbf{7} \\
				\hline
				e & \textbf{0} \\
				\hline
				f & 6 & \textbf{6} \\
				\hline
			\end{tabular}
		\end{center}

		\noindent In our first column, we knew the distances from $e$ to it's
		adjacent vertices. The blank cells are infinities unless there are
		previous cells. The shortest distance in that column is 0 for $e$ so we
		know the shortest path from $e$ to $e$ is 0. We take note at the top of
		the next column and update the paths to all reachable vertices. Then we
		pick the next shortest path which is 6 for $f$. One important thing to
		note is that we actually found a shorter path to $b$ by going through
		$f$. So we just fill out the table until there are no more paths to look
		at.

		\begin{center}
			\begin{tabular}{|c|c|c|c|c|c|c|c|c|c|c|}
				\hline
				\_ & \_ & e & f & d & a & b & c & i & h & g \\
				\hline
				a & & & & \textbf{12} \\
				\hline
				b & 20 & 20 & 20 & 17 & \textbf{17} \\
				\hline
				c & & & & 19 & 19 & \textbf{19} \\
				\hline
				d & 7 & 7 & \textbf{7} \\
				\hline
				e & \textbf{0} \\
				\hline
				f & 6 & \textbf{6} \\
				\hline
				g & & & & & & & & & \textbf{44} \\
				\hline
				h & & & & & & & & \textbf{32} \\
				\hline
				i & & & & & & & \textbf{29} \\
				\hline
			\end{tabular}
		\end{center}

		\noindent And finally we have the shortest distances from e to everything
		else (they're the bold numbers). We can also use this table to figure
		out the actual path, not just the distance. We know the distance of
		the shortest path from $e$ to $g$ is 44, but what is the actual path?
		Look on the table where the bolded number first appeared in its row.
		Notice at the top of that column is $h$. This means $h$ was visited
		before $g$. We repeat this easily and end up with $e, d, c, i, h, g$.

	\pagebreak

	\subsection{Floyd-Warshall Algorithm}
		The Floyd Warshall Algorithm is an algorithm for finding the shortest
		paths between all pairs of vertices. It relies on one idea which is
		that for every path between vertices $u$ and $v$, there may be a path that
		goes through $w$ such that
		\begin{center}
			\begin{tikzpicture}[nodes={draw, circle}, ->]
				\tikzset{edge/.style={midway,above,draw=none}}
				\node[draw] at (0, 0) (u){u};
				\node[draw] at (4, 0) (v){v};
				\node[draw] at (2, 1) (w){w};

				\draw[-] (u) -- (v);
				\draw[-] (u) -- (w);
				\draw[-] (v) -- (w);
			\end{tikzpicture} \\
			$dist(\{u, w\}) + (\{w, v\}) < dist(\{u, v\})$
		\end{center}

		\noindent Consider the following graph:

		\begin{center}
			\begin{tikzpicture}[nodes={draw, circle}, ->]
				\tikzset{edge/.style={midway,above,draw=none}}
				\node[draw] at (0, 0) (a){a};
				\node[draw] at (2, 0) (b){b};
				\node[draw] at (0, 2) (c){c};
				\node[draw] at (2, 2) (d){d};

				\draw[-] (a) -- (c) node[edge,left]{4};
				\draw[-] (a) -- (b) node[edge,bottom]{1};
				\draw[-] (b) -- (c) node[edge,right]{1.5};
				\draw[-] (b) -- (d) node[edge,right]{1};
				\draw[-] (c) -- (d) node[edge]{1};
			\end{tikzpicture}
		\end{center}

		\noindent We start out by filling a table with the known edges and
		distances along with another table that lists the initial vertex in the
		path.

		\begin{center}
			\begin{tabular}{cc}
				\begin{tabular}{|c|c|c|c|c|}
					\hline
					\_ & a & b & c & d \\
					\hline
					a & 0 & 1 & 4 & \\
					\hline
					b & 1 & 0 & 1.5 & 1 \\
					\hline
					c & 4 & 1.5 & 0 & 1 \\
					\hline
					d & & 1 & 1 & 0 \\
					\hline
				\end{tabular} &
				\begin{tabular}{|c|c|c|c|c|}
					\hline
					\_ & a & b & c & d \\
					\hline
					a & a & a & a & \\
					\hline
					b & b & b & b & b \\
					\hline
					c & c & c & c & c \\
					\hline
					d & & d & d & d \\
					\hline
				\end{tabular}
			\end{tabular}
		\end{center}

		\noindent And all we do is loop through all the possible vertices and
		treat them as potential $w$'s. For our first iteration, we will treat
		a as $w$. $u$ and $v$ will be the column and row of the path. Let's
		take an example: b as $u$ and c as $v$.

		\begin{center}
			\begin{tikzpicture}[nodes={draw, circle}, ->]
				\tikzset{edge/.style={midway,above,draw=none}}
				\node[draw] at (0, 0) (u){b};
				\node[draw] at (4, 0) (v){c};
				\node[draw] at (2, 1) (w){a};

				\draw[-] (u) -- (v) node[edge,below]{1.5};
				\draw[-] (u) -- (w) node[edge]{1};
				\draw[-] (v) -- (w) node[edge]{4};
			\end{tikzpicture}
		\end{center}

		\noindent As we can see, the distance is not shorter. However, we will
		check all pairs $(u, v)$ along with a as $w$. One example with this graph
		where it will be replaced is when b is $w$, and $u$ and $v$ are a and d
		respectively; let's take a look.

		\begin{center}
			\begin{tikzpicture}[nodes={draw, circle}, ->]
				\tikzset{edge/.style={midway,above,draw=none}}
				\node[draw] at (0, 0) (u){a};
				\node[draw] at (4, 0) (v){d};
				\node[draw] at (2, 1) (w){b};

				\draw[-] (u) -- (v) node[edge,below]{inf};
				\draw[-] (u) -- (w) node[edge]{1};
				\draw[-] (v) -- (w) node[edge]{1};
			\end{tikzpicture}
		\end{center}

		\noindent Naturally 2 is less than infinity so we would update our table
		with the value 2. We would also come across the same case when d is $u$
		and a is $v$ so I'll go ahead and insert 2 in that spot too. We would
		also update our second table with whatever value $w$ was, in this case
		b

		\begin{center}
			\begin{tabular}{cc}
				\begin{tabular}{|c|c|c|c|c|}
					\hline
					\_ & a & b & c & d \\
					\hline
					a & 0 & 1 & 4 & \textbf{2} \\
					\hline
					b & 1 & 0 & 1.5 & 1 \\
					\hline
					c & 4 & 1.5 & 0 & 1 \\
					\hline
					d & \textbf{2} & 1 & 1 & 0 \\
					\hline
				\end{tabular} &
				\begin{tabular}{|c|c|c|c|c|}
					\hline
					\_ & a & b & c & d \\
					\hline
					a & a & a & a & \textbf{b} \\
					\hline
					b & b & b & b & b \\
					\hline
					c & c & c & c & c \\
					\hline
					d & \textbf{b} & d & d & d \\
					\hline
				\end{tabular}
			\end{tabular}
		\end{center}

		\noindent We'll loop through all possible $u, v, w$ triplets and end up
		with the distances of shortest paths between all pairs of vertices in
		addition to a table that will help us figure out what the path actually
		is.

	\pagebreak

	\subsection{Minimum Spanning Trees (Prim's/Kruskal's Algorithm)}
		A minimum spanning tree is a subset of all edges of the graph that
		form a tree such that the total weight of the edges is minimized. \\

		\noindent There are two simple algorithms that do this: \\
		Prim's algorithm: take the shortest edge that adds a new node to the tree.
		Kruskal's algorithm: take the shortest edge that connects two unconnected
		trees. \\

		\noindent These are very simple and should pose no problem.
\end{document}
